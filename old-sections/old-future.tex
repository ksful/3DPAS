%the aim of the next subsection.

% What is quite noticeable is that the aspects of dynamism and
% distribution change even within a single application, and so can
% invalidate prior assumptions. \katznote{I don't think I understand the
%   point the previous sentence is trying to make.}
 
% The techniques developed are being applied ever more
% widely, both in terms of the application domains taking advantage of
% copious computing power and in terms of the volumes of data being
% processed.   

% so any
% reusable infrastructure will need to support a diversity of approaches
% if it is to gain broad applicability.


% : aside from the widespread (but far from ubiquitous) use of
% workflows,
% \jhanote{Rather than Science and Trends, I propose ``D3 Science
%   Trends'' -- maybe 1 trend along each ``D''? Then we could do some
%   similar prognostication for infrastructure and programming systems,
%   which would reflect the structure of the document?}




% Data science seems to be positioned in the same manner as network
% science:\jhanote{will network science be obvious? network -- as in
%   physical/layer network?} \katznote{I also have no idea what `network
%   science' means here.} a field that draws upon several disciplines and
% impacts upon many more, without actually being part of any single
% recognized speciality. \jhanote{What about the mouthful term: CDS\&E
%   -- computational and data-enabled science and engineering?}


% Scientific codes tend to be long-lived, and it therefore makes sense
% to design them for maintenance, for evolution, and specifically to
% facilitate the introduction of dynamic aspects over their
% lifetime. Software engineers have developed a host of techniques by
% which this can be accomplished, but these techniques need broader
% visibility---or at least broader application---in scientific
% computing.

\subsubsection{D3 Applications: Development and Software:} 
%\katznote{what is the title of this subsection? see latex source for
%  confusion}\jhanote{fixed} 

Dynamism and distribution are not unique to data science, and many of
the techniques developed in enterprise settings may usefully be
deployed.  Autonomic computing and communications can enormously
improve the utilization of physical infrastructure, and can contribute
to supporting dynamism. Autonomic principles can be applied at the
front end (in sensor networks) and the back end (in processing and
storage capabilities) of scientific infrastructure. \katznote{do we
  have examples of this being done we can refer to?}\jhanote{as
  mentioend, this is a candidate for en masse removal} While the
techniques developed for the latter are often not a perfect fit for
scientific applications, being transactional rather than
stream-oriented or stateful, there are certainly significant
opportunities both for deployment and for further research.  The
deployment of autonomic control is made easier if the application code
is structured as components, for example within a workflow, where
there are substantial opportunities for management interventions in
moving, replicating and otherwise adjusting the code ``in flight''.
\jhanote{The above paragraph is shaky and a candidate for removal if
  it cannot be linked with specific application examples or other
  sections of the paper}



% \jhanote{Above are fair points. We should expand this sub-section to
%   mirror Section 4 more closely}

% %\subsubsection{D3 Best Practices}
% \jhanote{I propose we move a discussion of autonomic computing out of
%   Infrastructure section and into best practises section, along with
%   S/W engineering etc.}


% \katznote{I wonder if this section is based on the evidence of this
%   paper, or more on what we think should be done.  I'm uncomfortable
%   with it, at least the first 2 paragraphs.  If nothing else, we
%   shouldn't call this Best Practices---rather, it is ``our''
%   opinions---but overall, I don't think it's the right way to go.}

% \jhanote{I echo Dan's comment that it is unclear whether this follows
%   from the paper, or if these are our (manufactured?) opinions?. There
%   is a need for citation to relevant sections/text of the paper.}

The growth in data science also suggests an elongation of the
lifetimes of both the code and data concerned, in the sense that data
is collected and analyzed with regard to a large-scale and often
long-running experimental program that may make its data widely
available. \katznote{Is the previous a general statement?  Or is it
  specific to something?  I don't think I understand what point is
  being made.}\jhanote{once again, this is entire paragraph is a
  candidate for en-masse removal, hence not responding to specific
  comments within the paragraph.} These developments change the game
for software development and maintenance. The use of proper software
engineering practices---architecture, design, reflection,
componentization, documentation, end-to-end engineering---become
vitally important in dealing with more dynamic scientific
domains. While these considerations can be avoided for single
applications, and often are seen as overheads to development and
experimentation, research in software engineering practice leaves no
room to doubt their effectiveness in constructing and (above all)
maintaining long-lived systems. It is important to remember the maxim
that a system that is not being changed is simply not being used, and
its corollary that a code will spent enormously more of its lifetime
being maintained and evolved than it spent being
developed. \katznote{does this previous maxim make sense here?  Even
  it so, do we need to state it as a maxim? Didn't we say this at the
  end of 6.3.1 already, anyhow?}  \jhanote{Is the maxim even correct?}
Maximizing our ability to manage this evolution makes good economic
and scientific sense.  \jhanote{Candidate to be removed entirely}

The components of scientific computations need themselves to be
managed, individually and collectively. Autonomic computing can
provide this, \emph{if} the relevant monitoring and control points are
available. The telecommunications industry has a long history of
constructing ``managed components'' that export independent functional
and management/monitoring interfaces. While this idea has not gained
enormous traction \katznote{does this mean almost no one uses it?}  in
much of the IT industry, it offers a way of integrating better, more
flexible management and adaptation to dynamism for scientific
applications, and frees the individual component developer from
worrying about global adaptation and decision-making issues. Dividing
the landscape in this way, separating scientific from management
functionalities, will help the development of re-usable
platforms. \jhanote{Again previous paragraph has to be linked to
  specific application and gaps.} \jhanote{Dan, more helpful than
  adding new comments, would be if you could help integrate as per
  existing comments. Else its already flagged as a deletion candidate
  and unlikely I will address new/different comments. Thanks!}

% \jhanote{propose we merge this paragraph with above
%   (industrial/enterprise developments) and the paragraph on autonomic
%   computing into a couple of sentences saying best practises abound in
%   specific contexts..}

% Scientific applications are not alone in having extensive data
% requirements: no matter to what extent this may have been true in the
% past, modern e-commerce generates data at similar rates and volumes as
% many scientific domains. This means that the techniques developed in
% industry to address these issues may be applicable to scientific
% codes, albeit with some modifications to suit different computational
% patterns, privacy constraints and other features.  Cloud computing and
% web services are both prime examples, and their use massively
% simplifies integration with other commodity services.

The rapid adoption of cloud computing in many industries perhaps masks
some of its limitations. Current cloud architectures are quite basic,
often either requiring very specific programming techniques or simply
providing low-level virtual machine management. The former is often
targeted at transactional, stateless computations that are a poor fit
to many scientific applications. The latter provide a certain degree
of elasticity but often provide interfaces without guidance as to
\emph{when} to change resources, or provide management consoles that
cannot be automated within an autonomic workflow. Nevertheless we can
confidently expect cloud computing to address these issues in time,
and further engagement of the scientific community with the providers
and researchers involved would help guide its
evolution. \katznote{again, do we have examples that we can point to
  in this section?}  \jhanote{We could have a bit of a discussion on
  how cloud computing (an increasingly important infrastructure issue)
  will influence D3 science?}


One technology that remains under-used \katznote{Again I would say
  that this is more opinion than evidence supported} is the semantic
web and the associated ideas of linked open data. While these might
seem like ideal technologies for scientific systems, they remain
marginalized. In many cases this may be due to unfamiliarity, the need
for quite sophisticated data handling, and the lack of decent
tooling. However scientific data retains many aspects of privacy,
sharing, and confidentiality that must be respected, and that can have
complex geographic, collaborative, organizational, and temporal
aspects. If these issues are handled in an integrated way---for
example by being specified as part of the overall experimental
workflow---then one might reasonably hope that linked open data would
be more widely adopted. \katznote{I again wonder if we are saying what
  we think should happen here}\jhanote{Yes, I agree. Either we relate
  this back to the applications or reduce in scope}

We also need to recognize the limitations on the reuse of scientific
codes. Aside from some common processing and visualization steps, much
code is too specific to be widely reused across domains (or even
within them). It is important not to overly prioritize the reuse of
components into unsuitable domains. \katznote{here I wonder if we
  should ask why there is so little reuse and suggest how there could
  be more.} \jhanote{also what aspects of the above are unique to
  3DPAS is important; candidate to go too}

